\chapter{Marco Tecnológico}
En el presente capítulo se muestran las herramientas necesarias para el desarrollo del proyecto, de tal manera que el lector pueda comprender los conceptos y tecnologías asociadas con la elaboración del mismo. 

\section{Lenguajes}
\subsection{C\#}
C\# es un lenguaje de programación desarrollado por Microsoft introducido en el año 2002, tiene seguridad de tipos y es orientado a objetos, el cual que permite a los desarrolladores crear una variedad de aplicaciones seguras y robustas que son ejecutadas en .NET Framework. Se puede usar C\# para crear aplicaciones cliente de Windows, servicios CML, componentes distribuidos, aplicaciones de bases de datos, aplicaciones cliente-servidor como es el caso de CPI (la cual usa este lenguaje para desarrollar el código del \textit{back-end}) , entre otras. \cite{cSharpMicrosoft}

\subsection{HTML}
HTML por sus siglas en inglés \textit{Hypertext Markup Language} , es el lenguaje de marcado central de la \textit{World Wide Web} que es la red informática mundial. Originalmente, HTML fue diseñado principalmente para describir semánticamente documentos científicos. Sin embargo, la generalidad de su diseño ha permitido que se adapte, en la actualidad, para describir otros tipos de documentos e incluso aplicaciones. \cite{htmlW3C}

Todas las vistas de la aplicación CPI fueron definidas por este lenguaje.


\subsection{CSS}
CSS por sus siglas en inglés \textit{Cascading Style Sheets} (Hojas de Estilo en Cascada), es el lenguaje para describir la presentación de páginas web (colores, diseños, fuentes, etc), permite a los usuarios agregar estilos a documentos estructurados como HTML. Al permitir la separacion del estilo de presentacion del contenido de los documentos, se facilita el mantenimiento de los sitios, el intercambio de hojas de estilo entre páginas y la personalización  de páginas en diferentes entornos.\cite{cssW3C}

\subsection{JavaScript}
JavaScript es un lenguaje de programación  interpretado conmunmente utilizado para el desarrollo web. Originalmente fue desarrollado por Netscape para crear contenido dinámico, control de multimedia, animación de imágenes, entre otras cosas a los sitios web.

Es un lenguaje de \textit{scripting} del lado del cliente, lo que significa que el código fuente es procesado por el navegador web del cliente y no en el servidor web. \cite{javaScriptChristensson}


\section{Entorno de trabajo}
\subsection{.NET}
\subsection{ASP.NET}
\subsection{ASP.NET MVC}
\subsection{ASP.NET Web Api}
\subsection{ASP.NET Razor}
\subsection{Umbraco}

\section{}
\subsection{SQL Server}
\subsection{JSON}
\subsection{Ajax}
\subsection{JQuery}
\subsection{DataTables}

