\chapter{Resultados}
Se cumplió el objetivo que se planteó como el alcance inicial de la versión 2 de la aplicación. Durante el desarrollo se consiguió un error en la estructura que se llevaba en la solución de Visual Studio ya que se estaban mezclando los controladores WebApi que se encargan de traer los datos necesarios para las diferentes funcionalidades, con los SurfaceController que se encargan de las vistas por lo que se tomó un tiempo en corregir este error y el tiempo para completar las funcionalidades que quedaban fue mucho mas justo.

Aunque se tuvo una fase de inducción, el pasante a lo largo del desarrollo del proyecto tuvo que mantenerse en constante aprendizaje de conceptos nuevos que aparecían durante el camino y que eran necesarios entender para poder completar el mismo, por lo que en algunos casos relantizó el desarrollo. El caso de uso que no se logró completar pertenecía al Módulo de Generación de Reportes. 


\section{Estado actual de la aplicación CPI}
La aplicación CPI tuvo como resultado una versión 2 funcional que abarca el 97\% de los casos de uso planteados en el alcance inicial. Al final del trabajo realizado la empresa quedó satisfecha con el trabajo realizado por el pasante y el resultado final de la aplicación. En la Tabla \ref{tab:casosDeUso} se muestran los casos de uso del sistema.



\newcounter{magicrownumbers}
\newcommand\rownumber{\stepcounter{magicrownumbers}\arabic{magicrownumbers}}
\begin{center}
    \begin{longtable}{ | l | l | c | }
        \hline
        \rowcolor{blue!25}
        \multicolumn{1}{|c|}{ID del Caso de Uso} &
        \multicolumn{1}{|c|}{Caso de Uso} &
        \multicolumn{1}{|c|}{Actor} \\
        \hhline{===}
        \endhead

        \endfoot

       CU-\rownumber & Iniciar sesión (Umbraco) & Admin \\ \hline
       CU-\rownumber & Consultar lista miembros & Admin \\ \hline
       CU-\rownumber & Gestionar miembro (CRUD) & Admin \\ \hline
       CU-\rownumber & Consultar grupos & Admin \\ \hline
       CU-\rownumber & Gestionar grupo (CRUD) & Admin \\ \hline
       CU-\rownumber & Asignar miembro/s a grupo & Admin \\ \hline
       CU-\rownumber & Remover miembro/s de grupo (CRUD) & Admin \\ \hline
       CU-\rownumber & Cambiar permisos de miembro & Admin \\ \hline

       CU-\rownumber & Gestionar contenido & Admin \\ \hline
       CU-\rownumber & Consultar lista de cadenas & Admin \\ \hline
       CU-\rownumber & Gestionar cadena (CRUD) & Admin \\ \hline
       CU-\rownumber & Asignar tienda/s a cadena & Admin \\ \hline
       CU-\rownumber & Remover tienda/s de cadena & Admin \\ \hline
       CU-\rownumber & Consultar lista de productos & Admin \\ \hline
       CU-\rownumber & Gestionar producto (CRUD) & Admin \\ \hline
       CU-\rownumber & Consultar lista de productos UPC & Admin \\ \hline
       CU-\rownumber & Gestionar producto UPC (CRUD) & Admin \\ \hline
       CU-\rownumber & Asignar producto/s a producto UPC & Admin \\ \hline
       CU-\rownumber & Remover producto/s de producto UPC & Admin \\ \hline
       CU-\rownumber & Consultar lista de precios & Admin \\ \hline
       CU-\rownumber & Consultar lista de tiendas & Admin \\ \hline
       CU-\rownumber & Gestionar tienda (CRUD) & Admin \\ \hline
       CU-\rownumber & Consultar lista de zonas & Admin \\ \hline
       CU-\rownumber & Gestionar zonas (CRUD) & Admin \\ \hline
       CU-\rownumber & Consultar lista de catálogo generales & Admin \\ \hline
       CU-\rownumber & Gestionar catálogo (CRUD) & Admin \\ \hline
       CU-\rownumber & Asignar producto/s a catálogo & Admin \\ \hline
       CU-\rownumber & Remover producto/s de catálogo & Admin \\ \hline

       
       CU-\rownumber & Iniciar sesión (CPI) & Retailer \\ \hline
       CU-\rownumber & Consultar lista de productos & Retailer \\ \hline
       CU-\rownumber & Importar precios desde archivo excel & Retailer \\ \hline
       CU-\rownumber & Consultar reporte precio producto & Retailer \\ \hline
       CU-\rownumber & Exportar reporte precio producto & Retailer \\ \hline
       CU-\rownumber & Filtrar reporte precio producto por fecha & Retailer \\ \hline
       CU-\rownumber & Consultar reporte precio histórico de producto & Retailer \\ \hline
       CU-\rownumber & Filtrar reporte precio histórico por fechas & Retailer \\ \hline
       CU-\rownumber & Consultar reporte gráfico dispersión por catálogo & Retailer \\ \hline
       CU-\rownumber & Exportar reporte gráfico dispersión por catálogo & Retailer \\ \hline
       CU-\rownumber & Filtrar gráfico dispersión por catálogo & Retailer \\ \hline
       CU-\rownumber & Filtrar gráfico dispersión por fecha & Retailer \\ \hline
       CU-\rownumber & Consultar lista de catálogos personalizados & Retailer \\ \hline

       \caption{Resumen de casos de uso de la aplicación web CPI.}
       \label{tab:casosDeUso}
       

    \end{longtable}
\end{center}

La nueva aplicación desarrollada funciona como un agregado a sitios de Umbraco. No se ha desplegado en un ambiente de producción, solo fue probada en varias instalaciones locales, a pesar de esto el dueño del producto quedó satisfecho con los resultados obtenidos. Sigue un resumen de la funcionalidad por módulos desarrollada, dividido en partes: funcionalidad en el front end de eFuel y funcionalidad en el back office de Umbraco.

\subsection{Funcionalidad en el front end}
\begin{itemize}
    \item \emph{Seguridad}: Inicio de sesión con credenciales.
    
    \item \emph{Productos}: Listado de productos con filtros para el producto UPC y descripción del producto. 

    \item \emph{Reportes}: Generación de reportes de:
            \begin{itemize}
                 \item Precio producto, para comparar los precios del mismo productos entre las cadenas que son competencia.
                 \item Precios históricos de un producto en un rango de fechas, para ver el cambio del precio de un producto.
                 \item Precios de productos en un catálogo.
            \end{itemize}

          Además exportar los reportes que poseen tablas a un archivo en formato Microsoft Excel.
 
\end{itemize}

\subsection{Funcionalidad en el back office de Umbraco}
\begin{itemize}
    \item \emph{Seguridad}: inicio de sesión con credenciales; administración (creación, edición, detalles, eliminación y permisos) de miembros de CPI.
    
    \item \emph{Cadenas}: listado de cadenas; administración (creación, edición, vista de detalles y eliminación) de cadenas.
    
    \item \emph{Zonas}: listado de zonas; administración (creación, edición, vista de detalles y eliminación) de zonas.
    
    \item \emph{Administración}: listado de productos, precios, tiendas, y catálogos; administración (creación, edición, vista de detalles y eliminación) de productos, precios, tiendas, y catálogos. 
\end{itemize}
