\chapter*{Introducción}
iKêls Consulting \cite{ikels} es una empresa especializada en el desarrollo de aplicaciones y sitios web dedicada al sector corporativo utilizando Umbraco como la plataforma principal de manejo de contenido (CMS) \cite{cmsBarker}. Umbraco es una herramienta de gestión de contenido de código abierto desarrollado sobre ASP.NET \cite{aspMicrosoft} para sistemas Windows de Microsoft y usando el lenguaje de programación C\# \cite{cSharpMicrosoft}, cuenta con una gran aceptación a nivel mundial y una amplia comunidad de desarrolladores.

\section*{Antecedentes}
 iKêls Consulting \cite{ikels} posee entre sus activos una aplicación web llamada CPI que permite realizar comparaciones de precios entre productos ofrecidos por diferentes tiendas de ventas al detal, sin embargo esta aplicación se desarrolló hace más de 10 años usando tecnología Classic ASP y VBScript por lo que la empresa necesita realizar un proceso completo de re-ingeniería para adaptar esta solución a plataformas y modelos de gestión más modernos.


\section*{Planteamiento del problema}
Las cadenas de supermercados operan en un entorno comercial muy competitivo con márgenes muy pequeños de rentabilidad, por lo que para poder diferenciarse del resto de opciones en el mercado, deben confeccionar una estructura de precios que resulte simultáneamente atractiva para atraer a sus clientes así como rentable para mantener el negocio a flote.
Para lograr lo anterior es imprescindible conocer los rangos dónde se mueve el precio de cada producto de manera constante. Debido al volumen de datos involucrados (miles de productos de varios competidores) se necesita una aplicación que permita consolidar y analizar la información de manera rápida, sencilla y efectiva.
\section*{Justificación e importancia}
Los departamentos de análisis de precios de las cadenas minoristas invierten la mayor parte de sus recursos en actividades de investigación.
Al disponer de una herramienta que facilite su trabajo permitirá responder más rápidamente ante los cambios del mercado así como cubrir un mayor volumen de productos en un menor  tiempo, ya que  poder identificar de manera oportuna aquellos productos donde se requiere ajustes de precios (ya sea para competir con otras cadenas o para no perder dinero), puede ser el factor clave para definir los resultados netos de la empresa.
\section*{Objetivo general}
Crear una nueva versión de la aplicación web CPI integrada con la plataforma de manejo de contenido Umbraco para el ingreso y administración de precios de productos.
\section*{Objetivos específicos}
\begin{itemize}
   \item Crear el módulo de administración de cadenas, productos y categorías.
   \begin{itemize}
       \item Desarrollar interfaz para creación, edición y eliminación de registros.
       \item Asignar productos a categorías
       \item Listar productos con mecanismo de búsqueda.
   \end{itemize}

   \item Crear el módulo de administración de precios.
   \begin{itemize}
       \item Desarrollar interfaz para creación, edición y eliminación de los registros de precios.
       \item Importar un lote de precios usando un archivo en formato Microsoft Excel
   \end{itemize}

   \item Crear el módulo de generación de reportes.
   \begin{itemize}
       \item Crear reporte para comparar precios de productos por categorías
       \item Crear reporte comparativo entre precios de productos de diferentes cadenas.
       \item Crear reporte histórico de precios de un producto
   \end{itemize}
\end{itemize}
