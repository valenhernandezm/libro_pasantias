\chapter*{Introducción}
iKêls Consulting es una empresa que se especializa en el desarrollo de aplicaciones y sitios web para el sector corporativo utilizando Umbraco como la plataforma principal de manejo de contenido (CMS). Umbraco es una herramienta de código abierto que se desarrollo en Dinamarca usando el lenguaje de programación C\# y ASP.NET para sistemas Windows de Microsoft que cuenta con una gran aceptación a nivel mundial y una amplia comunidad de desarolladores. La empresa posee entre sus activos una aplicación web llamada CPI que permite realizar comparaciones de precios entre productos ofrecidos por diferentes tiendas de vendas al detal. Esta aplicación aplicación se desarrolló hace mas de 10 años usando tecnología Classic ASP y VBScript.

\section*{Planteamiento del problema}

\section*{Justificación e importancia}

\section*{Objetivo general}

\section*{Objetivos específicos}
\begin{itemize}
    \item Crear el módulo de administración de clientes, productos, precios, vehículos, rutas y conductores.
    \begin{itemize}
        \item Creación, edición y eliminación de registros.
        \item Calcular costo de cisterna según volumen y precios de productos.
        \item Asignación de vehículos y tipos de producto a clientes.
        \item Listado de registros con mecanismo de búsqueda.
    \end{itemize}

    \item Crear el módulo de administración de pedidos.
    \begin{itemize}
        \item Crear, editar y eliminar pedidos mediante consola especializada.
        \item Opciones diferenciadas según perfil del usuario.
        \item Listado para consulta de pedidos con opción de exportar en formato MS Excel.
    \end{itemize}

    \item Crear el módulo de conciliación de pagos.
    \begin{itemize}
        \item Mecanismo para registro de pagos totales o parciales.
        \item Mecanismo de conciliación y asignación de pagos a pedidos.
    \end{itemize}

    \item Crear el módulo de registro de usuarios y perfiles.
    \begin{itemize}
        \item Registro y edición de usuarios con perfil de permisos.
        \item Asociar usuarios a clientes.
    \end{itemize}
\end{itemize}