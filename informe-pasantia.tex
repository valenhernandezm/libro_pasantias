% Centro de Estadística y Matemática Aplicada
% Universidad Simón Bolívar
% Plantilla LaTeX para manuscritos (tesis y pasantías)
% pregrado y postgrado
%
% Andrés M. Sajo-Castelli
% Carlos Contreras
%
% 15 Abril 2015 -- primera versión pública
% ...
% 11 Mayo 2018 --- Se agrega bibliografía en castellano via babelbib
%
\documentclass[pregrado]{tesis-usb}

% paquetes
\usepackage[utf8]{inputenc}
\usepackage[utf8]{inputenc}
\usepackage{ragged2e}
\usepackage{indentfirst}
\usepackage{longtable}
\usepackage{tabu}
\usepackage{float}
\usepackage{hhline}
\usepackage{makecell}
\usepackage{multirow}
\usepackage{array}
\usepackage{verbatim}
\usepackage{acronym}
\usepackage{amsmath}
\usepackage{amsfonts}
\usepackage{amssymb}
\usepackage{float}
\usepackage{pdfpages}
\usepackage{xcolor}
\definecolor{red}{HTML}{FF4C4C}
% \usepackage{hyperref}

% estilo de las referencias
\usepackage[fixlanguage]{babelbib-and}\selectbiblanguage{spanish}
\usepackage{url}
\bibliographystyle{IEEEtran}

\usepackage{graphicx}
% \graphicspath{ {../imgs/} }

\autor{Valentina Hernández}
\autori{V. Hernández}
\usbid{10-10353}
\titulo{Desarrollo de la versión 2 de la aplicación web CPI}
\fecha{Septiembre~de~2018}
\agno{2018}
\fechadefensa{30~de~noviembre~de~2018}
\tutor{Tutor Académico: Prof. Soraya Carrasquel}
\usarcotutor
\cotutor{Tutor Industrial: Ing. José Cerqueiro}
\trabajo{Informe de Pasantía}
\coord{Ingeniería de la Computación}
\grado{Ingeniero de la Computación}
\carrera{Ingenieria de la Computación}
\programa{Nombre del Programa}
\juradouno{Nombre y Apellido}
\juradodos{Nombre y Apellido}
\juradotres{Nombre y Apellido}

% Cambia comillas simple por comilla cerrada en ambiente verbatim
\makeatletter
\let \@sverbatim \@verbatim
\def \@verbatim {\@sverbatim \verbatimplus}
{\catcode`'=13 \gdef \verbatimplus{\catcode`'=13 \chardef '=13 }}
\makeatother

\begin{document}

\frontmatter
\maketitle
\begin{resumen}
     Es una exposici\'on clara del tema tratado en el trabajo, de los objetivos, de la metodolog\'ia utilizada, de los resultados relevantes obtenidos y de las conclusiones. Mismo tipo de fuente seleccionado con tamaño 12 e interlineado sencillo en el p\'arrafo. El resumen no debe exceder de trescientas (300) palabras escritas. \\
     Palabras cl\'aves: palabras, cl\'aves, separadas por coma, cinco m\'aximo.
\end{resumen}
\tableofcontents
\listoffigures

\mainmatter
\chapter*{Introducción}
\par Introducción aquí.
\chapter{Entorno empresarial}
En este capítulo se presenta una descripción del entorno en el cual se desarrolló el proyecto de pasantía en la empresa iKêls Consulting. Comprende una breve reseña histórica, su misión y visión, la estructura organizacional y el área a la cual el pasante estuvo asignado.

\section{Antecedentes de la empresa}
IKêls Consulting se creó el año 2008 como una empresa dedicada al desarrollo de soluciones en el área de Sistemas de Información. Los fundadores contaban con una amplia trayectoria en los procesos y tecnología para la elaboración de documentación técnica avanzada (por ejemplo normas ISO para construcción de plantas petroquímicas).

Para aprovechar la experiencia previa los productos y servicios se concentran en el área de aplicaciones web (por ejemplo, sistemas de manejo de contenido o CMS) para el sector corporativo atendiendo a un selecto grupo de clientes con presencia local e internacional.

Actualmente, las actividades principales se concentran en:

\begin{itemize}
  \item Construcción de portales web en múltiples idiomas y que pueden ser administrados por sus propios dueños. Esto incluye la programación de módulos especiales para integrar información desde y hacia sistemas externos, desplegar datos de manera amigable o generar notificaciones automáticas dependientes de actividades de los visitantes u otros eventos.
  \item Apoyo en la gestión de contenido de portales web.
  \item Consultoría y gestión para optimizar las variables asociadas al rendimiento y desempeño de las páginas web. Teniendo especial interés en el monitoreo de presencia en buscadores, evaluación del perfil de los visitantes y garantizar un nivel adecuado de usabilidad en diferentes dispositivos, etc.
  \item Desarrollo de productos personalizados que complementen las ventajas y facilidades de los dispositivos móviles en sincronización con mecanismos de soporte en servidores web.
  \item Desarrollo de soluciones especializadas para ofrecer bajo el modelo SaaS o Software as a Service.
\end{itemize}

\section{Misión}
Proveer productos y servicios en el área de sistemas de información que permitan una comunicación efectiva de nuestros clientes con su público y también sirva como plataforma de trabajo donde se aprovechen las innovaciones y ventajas de las tecnologías más modernas.

\section{Visión}
Deseamos ser un proveedor confiable, que ofrece un alto valor agregado en cada producto o servicio que prestamos a nuestros clientes.

\section{Ubicación del pasante}
El proyecto de pasantía pertenece al grupo de desarrollo de aplicaciones y cuenta con la dirección del Presidente de la Empresa y con el apoyo de los ingenieros líderes del grupo.

\section{Organigrama}
\begin{figure}[hbt]
\begin{center}
\includegraphics[scale=0.8]{organigrama}
\caption{Organigrama de la Empresa. Fuente: Elaboración propia.}
\label{fig:figura1}
\end{center}
\end{figure}

\chapter{Marco Teórico}
En el presente capítulo se definen las bases teóricas sobre las cuáles se apoya el proyecto.

\section{Bases Teóricas}

\subsection{CMS}
Un Sistema de Gestión de Contenido o CMS, por sus siglas en inglés \textit{Content Management System}, es un paquete de software que brinda cierto nivel de automatización de las tareas requeridas para administrar el contenido de manera efectiva. Un CMS permite a los usuarios crear nuevo contenido, editar contenido existente, y hacer el contenido accesible al público \cite{cmsBarker}.

Para los editores el CMS les permite crear contenido nuevo, editar contenido existente, realizar procesos en el contenido mediante una interfaz de edición (referida como \textit{back-end} que es la capa de acceso a los datos de la aplicación) y finalmente les permite colocar este contenido a disposición de otros usuarios en una interfaz (referida como el \textit{front-end}, es decir, la capa de presentación de la aplicación).

Un CMS permite el control del contenido, esto se refiere a que mantiene un constante seguimiento del contenido (donde se encuentra el contenido, quién puede acceder a él, y cómo se relaciona con otros contenidos). Por otro lado permite la reutilización del contenido (usar contenido en más de un lugar).

Una de las mayores ventajas de el uso de CMS es que facilita las tareas mencionadas anteriormente para usuarios que no tienen preparación técnica. En el caso de la aplicación CPI, la mayoría de los usuarios no poseen conocimientos especializados en el área de computación, por lo cual es conveniente el desarrollo del sistema sobre un CMS

El CMS sobre el cual se trabajó para el desarrollo de la aplicación cuenta con funcionalidades que ya están implementadas que son necesarias para ésta, como la autenticación para los usuarios, permisología, interfaces que facilitan el manejo de la base de datos, entre otras cosas. Cuenta con gran variedad de paquetes, librerías y módulos que facilitan el desarrollo de la aplicación. 

\subsection{Modelo Cliente-Servidor}
Arquitectura de redes de computadoras ampliamente utilizado y que forma la base del uso de redes en gran medida, consta de dos entidades: un cliente y un servidor. El cliente le envía una solicitud al servidor y espera una respuesta. Luego, el servidor recibe la solicitud, lleva a cabo el trabajo requerido, o busca los datos solicitados y devuelve una respuesta al cliente \cite{redesTanenbaum}.

El servidor mantiene una relación de uno-a-muchos con los clientes, por otro lado ambos términos pueden ser vistos como “roles”, pues es posible que una máquina o proceso ejecute labores tanto de cliente como de servidor (por ejemplo, un servidor puede enviar una petición a otro servidor, si el mismo carece de los recursos que le fueron solicitados, convirtiéndose así en un cliente). Es importante destacar que los términos “cliente” y “servidor” pueden referirse tanto a máquinas como a programas o procesos. Esta arquitectura es ampliamente utilizada en aplicaciones web.

\subsection{MVC}
MVC, siglas para Modelo-Vista-Controlador, es un patrón de software utilizado ampliamente en la actualidad. Consiste en separar los datos de la aplicación (Modelo), la interfaz con el usuario (Vista), y la lógica de control (Controlador) en tres componentes distintos. Al realizar esta separación se reduce la complejidad del diseño arquitectónico y se incrementa la flexibilidad, la reusabilidad y mantenimiento del código. Adicionalmente, se pueden realizar cambios sobre un componente sin afectar a los demás, lo cual permite que cada componente tenga ciclos de desarrollo independientes del resto. \cite{mvcKrasner}. 

Cabe destacar que este patrón es usado frecuentemente en el desarrollo de aplicaciones web, por lo que resulta bastante sencillo aplicarlo al modelo cliente-servidor utilizado en la aplicación. CPI fue desarrollado usando una plataforma de desarrollo web basada en .NET que implementa este patrón.

\subsection{API}
Un API, llamado así por sus siglas en inglés para \textit{Application Programming Interface}, es un conjunto de comandos, funciones, protocolos y objetos que permiten exponer los datos de una aplicación de software, establecen las reglas y los mecanismos a través de los cuales se puede tener acceso a estos datos. También permite la interacción entre con algún software externo.\cite{apiChristensson}

El API sirve como intermediario entre dos aplicaciones, por ejemplo una aplicación dispone del API de otra para obtener los datos que se encuentran en la última y usarlos para proveer algún servicio a sus usuarios. En el caso de CPI se desarrolló un API para poder acceder a datos de la aplicación.

\subsection{Servicio Web}
Un servicio web es una aplicación o fuente de datos a la que se puede acceder a través de un protocolo web estándar, diseñado para soportar interacción máquina-máquina a través de una red, proveen una vía estándar para la comunicación entre distintas aplicaciones de software ejecutadas en distintas plataformas y ambientes. La mayoría de los servicios web proporcionan un API, para que se puedan acceder a los datos. \cite{webServiceChristensson}

Para la aplicación CPI en el desarrollo se incluyó un módulo de servicios web, el cual permite el acceso a datos de la aplicación.

\chapter{Marco Tecnológico}
En el presente capítulo se muestran las herramientas necesarias para el desarrollo del proyecto, de tal manera que el lector pueda comprender los conceptos y tecnologías asociadas con la elaboración del mismo.

\section{Lenguajes}
\subsection{C\#}
\subsection{HTML}

\subsection{CSS}
\subsection{JavaScript}

\section{Entorno de trabajo}
\subsection{.NET}
\subsection{ASP.NET}
\subsection{ASP.NET MVC}
\subsection{ASP.NET Web Api}
\subsection{ASP.NET Razor}
\subsection{Umbraco}

\section{}
\subsection{SQL Server}
\subsection{JSON}
\subsection{Ajax}
\subsection{JQuery}
\subsection{DataTables}


\chapter{Marco Metodológico}
En este capítulo se explicará la metodología que se utilizó para el desarrollo del proyecto, que fue necesaria usar para cumplir con los objetivos planteados, conocida como Srum. A continuación se describirá el marco de desarrollo, las actividades y resultados de cada una de las etapas de esta metodología.

\section{¿Qué es Scrum?}
Scrum es un marco de trabajo para desarrollar, entregar y mantener productos complejos, en el cual las personas pueden abordar problemas complejos adaptativos, y a su vez entregar productos del máximo valor posible productiva y creativamente. Empezó a ser usado desde principios de los años 90. Scrum es un marco de trabajo dentro del cual se pueden emplear varios procesos y técnicas. El marco de trabajo Scrum consiste en los Equipos Scrum y sus roles, eventos, artefactos y reglas asociadas. \cite{scrumSchwaber}

Usando correctamente este marco de trabajo se puede mostrar la eficacia relativa de las técnicas de la gestión del producto y las técnicas de trabajo, de manera que a medida que va pasando el tiempo se pueda mejorar el producto, el equipo y el entorno de trabajo. Para que esto suceda es necesario que cada componente que interactúa dentro de Scrum cumpla con su propósito específico.

\section{Usos de Scrum}
Scrum inicialmente fue desarrollado para la gestión y desarrollo de productos. Se ha usado principalmente para:

\begin{enumerate}
	\item Investigar e identificar mercados viables, tecnologías y capacidades de productos.
	\item Desarrollar productos y mejoras.
	\item Liberar productos y mejoras tantas veces como sea posible durante el día.
	\item Desarrollar y mantener ambientes en la Nube (en línea, seguros, bajo demanda) y otros entornos operacionales para el uso de productos.
	\item Mantener y renovar productos.
\end{enumerate}

\section{Equipo de Scrum}

El equipo Scrum está conformado por el Dueño del Producto (\textit{Product Owner}), el Equipo de desarrollo (\textit{Development Team}) y un \textit{Scrum Master}. Los equipos Scrum son autoorganizados y multifuncionales. Los equipos autoorganizados eligen la mejor forma de llevar a cabo su trabajo y no son dirigidos por personas externas al equipo. Los equipos multifuncionales tienen todas las competencias necesarias para llevar a cabo el trabajo sin depender de otras personas que no son parte del equipo. El modelo de equipo en Scrum está diseñado para optimizar la flexibilidad, la creatividad y la productividad. \cite{scrumSchwaber}

\subsection{Dueño del Producto} \label{productOwner}
Es el responsable de maximizar el valor del producto resultante del trabajo del Equipo de Desarrollo. Es la única persona responsable de gestionar la Lista del Producto (\textit{Product Backlog}), esto incluye:
\begin{itemize}
\item Expresar claramente los elementos de la lista.
\item Ordenar los elementos de la lista, para alcanzar objetivos y misiones de manera eficiente.
	\item Optimizar el valor del trabajo que el Equipo de Desarrollo realiza.
	\item Asegurar que la lista sea visible, transparente y clara para todos y que muestra aquello en lo que el equipo trabajará a continuación.
	\item Asegurar que el Equipo de Desarrollo entiende los elementos de la Lista del Producto al nivel necesario.
\end{itemize}

\begin{itemize}
    \item Expresar los items del \emph{Product Backlog}
\end{itemize}

\subsection{Equipo de Desarrollo}
El Equipo de Desarrollo está conformado por profesionales que realizan el trabajo de entregar un Incremento (ver sección \ref{incremento}) de producto “Terminado” que potencialmente se puede poner en producción al final de cada Sprint (ver sección \ref{sprint}). Cabe destacar que solo los miembros del Equipo de Desarrollo participan en la creación del Incremento, ellos se organizan y gestionan su propio trabajo. Los Equipos de Desarrollo tienen las siguientes características: 

\begin{itemize}
	\item Son autoorganizados. Nadie (ni siquiera el Scrum Master) indica al Equipo de Desarrollo cómo convertir elementos de la Lista del Producto en Incrementos de funcionalidad potencialmente desplegables.
	\item Los Equipos de Desarrollo son multifuncionales, esto es, como equipo cuentan con todas las habilidades necesarias para crear un Incremento de producto.
	\item Scrum no reconoce títulos para los miembros de un Equipo de Desarrollo independientemente del trabajo que realice cada persona
	\item Los Miembros individuales del Equipo de Desarrollo pueden tener habilidades especializadas y áreas en las que estén más enfocados, pero la responsabilidad recae en el Equipo de Desarrollo como un todo. 
\end{itemize}


\subsection{Scrum Master} \label{scrumMaster}

El Facilitador (o \emph{Scrum Master}) es el responsable de promover y apoyar Scrum como está definido en la Guía de Scrum (ver \cite{scrumSchwaber}). Esto lo logran ayudando a todo el Equipo Scrum a entender la teoría, práctica, reglas y valores de Scrum. \cite{scrumSchwaber}

\section{Eventos de Scrum}
\subsection{La Iteración} \label{sprint}


\section{Artefactos de Scrum}
\subsection{Lista de Objetivos del Producto} \label{productBacklog}
La Lista de Objetivos del Producto es una lista ordenada de todas las tareas que deben ser completadas para entregar el producto. Esta lista nunca está completa, la primera versión muestra los primeros requerimientos conocidos y mejor entendidos. \cite{scrumSchwaber} La Lista de Objetivos del Producto evoluciona junto con el producto para actualizar los requerimientos dependiendo de las exigencias del cliente y del mercado. El Dueño del Producto (ver sección \ref{productOwner}) es el responsable de este artefacto.

\subsection{Lista de Tareas de la Iteración}
La Lista de Tareas de la Iteración es el conjunto de tareas de la lista de objetivos del producto seleccionadas para completar en una Iteración, puede verse como una predicción de lo que estará completado al llevarse a cabo la Iteración. \cite{scrumSchwaber} Esta lista va a ir cambiando a medida que el Equipo de Desarrollo identifique tareas y trabajo que debe ser realizado para completar los items durante el desarrollo de la Iteración, es decir, además de las tareas tomadas de la Lista de Objetivos del Producto, pueden surgir nuevas tareas que deben ser cumplidas para llevar a cabo el trabajo de la Iteración.

\subsection{Tablero de tareas}
El tablero de tareas es una tabla utilizada para gestionar el estado de los objetivos de la lista de objetivos del producto, contiene 4 columnas:

\begin{itemize}
   \item \textbf{Por hacer}: contiene las tareas que no se han empezado.
   \item \textbf{Haciendo}: contiene las tareas en las que se está trabajando actualmente.
   \item \textbf{Hechas}: contiene las tareas que se han completado.
   \item \textbf{Mejoras}: contiene tareas u objetivos que salen del alcance del proyecto actual y que pueden ser desarrolladas para una versión futura del producto.
\end{itemize}

Cada una de las tareas en este tablero puede ser asignada a una persona específica, también se pueden etiquetar dependiendo de la naturaleza de la tarea.

\subsection{Incremento} \label{incremento}



\chapter{Desarrollo}

\section{Fase I: Iniciación}
\subsection{Levantamiento de información}
\subsection{Plan de proyecto}
\subsection{Visión del Sistema}
Para esta sección hay que hacer un \textit{Documento de Visión del Sistema}.
\subsection{Lista Inicial de Requerimientos}
Para esta sección hay que hacer un \textit{Listado de Requerimientos Funcionales}.
\subsection{Modelo Inicial de Casos de Uso}
Para esta sección hay que hacer un \textit{Documento de Especificación de Requerimientos del Sistema}.
\subsection{Glosario del Sistema}
Para esta sección hay que hacer un \textit{Glosario del Sistema}.

\section{Fase II: Elaboración}
\subsection{Lista de Requerimientos Suplementarios}

\subsection{Descripción de la Arquitectura de Software}
\subsubsection{Vista Lógica}
\paragraph{Modelo Conceptual}
\paragraph{Diagrama de Clases}
\paragraph{Modelo Entidad-Relación}

\subsubsection{Vista de Implementación}
Aquí hay que hacer el \textit{Diagrama de Componentes del Sistema} para mostrar la separación del sistema en las capas de MVC.
\subsubsection{Vista de Implantación?}
Hay que saber si esta vista es relevante para el sistema.
\subsubsection{Vista de Procesos?}
La misma pregunta que para la sección anterior.
\subsubsection{Vista de Casos de Uso}
Aquí hay que hacer una lista con los casos de uso más importantes detallados.

\section{Fase III: Construcción}
Describir el proceso de desarrollo en pasos.

\section{Fase IV: Transición}
\subsection{Plan de Pruebas}
\chapter*{Conclusiones y Recomendaciones}
\par Conclusiones aquí.
\nocite{*}
\bibliography{referencias}
%\appendix

\end{document}
