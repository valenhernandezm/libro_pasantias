\chapter{Marco Metodológico}
En este capítulo se explicará la metodología que se utilizó para el desarrollo del proyecto, que fue necesaria usar para cumplir con los objetivos planteados, conocida como Scrum. A continuación se describirá el marco de desarrollo, las actividades y resultados de cada una de las etapas de esta metodología.

\section{Scrum}
Scrum es un marco de trabajo para desarrollar, entregar y mantener productos complejos, en el cual las personas pueden abordar problemas complejos adaptativos, y a su vez entregar productos del máximo valor posible productiva y creativamente. Empezó a ser usado desde principios de los años 90. Scrum es un marco de trabajo dentro del cual se pueden emplear varios procesos y técnicas. El marco de trabajo Scrum consiste en los Equipos Scrum y sus roles, eventos, artefactos y reglas asociadas \cite{scrumSchwaber}.

Usando correctamente este marco de trabajo se puede mostrar la eficacia relativa de las técnicas de la gestión del producto y las técnicas de trabajo, de manera que a medida que va pasando el tiempo se pueda mejorar el producto, el equipo y el entorno de trabajo. Para que esto suceda es necesario que cada componente que interactúa dentro de Scrum cumpla con su propósito específico.

\section{Usos de Scrum}
Scrum inicialmente fue desarrollado para la gestión y desarrollo de productos. Se ha usado principalmente para:

\begin{enumerate}
	\item Investigar e identificar mercados viables, tecnologías y capacidades de productos.
	\item Desarrollar productos y mejoras.
	\item Liberar productos y mejoras tantas veces como sea posible durante el día.
	\item Desarrollar y mantener ambientes en la Nube (en línea, seguros, bajo demanda) y otros entornos operacionales para el uso de productos.
	\item Mantener y renovar productos.
\end{enumerate}

\section{Equipo de Scrum}

El equipo Scrum está conformado por el Dueño del Producto (\textit{Product Owner}), el Equipo de desarrollo (\textit{Development Team}) y un \textit{Scrum Master}. Los equipos Scrum son autoorganizados y multifuncionales. Los equipos autoorganizados eligen la mejor forma de llevar a cabo su trabajo y no son dirigidos por personas externas al equipo. Los equipos multifuncionales tienen todas las competencias necesarias para llevar a cabo el trabajo sin depender de otras personas que no son parte del equipo. El modelo de equipo en Scrum está diseñado para optimizar la flexibilidad, la creatividad y la productividad \cite{scrumSchwaber}.

\subsection{Dueño del Producto} \label{productOwner}
Es el responsable de maximizar el valor del producto resultante del trabajo del Equipo de Desarrollo. Es la única persona responsable de gestionar la Lista del Producto (\textit{Product Backlog}), esto incluye:
\begin{itemize}
\item Expresar claramente los elementos de la lista.
\item Ordenar los elementos de la lista, para alcanzar objetivos y misiones de manera eficiente.
	\item Optimizar el valor del trabajo que el Equipo de Desarrollo realiza.
	\item Asegurar que la lista sea visible, transparente y clara para todos y que muestra aquello en lo que el equipo trabajará a continuación.
	\item Asegurar que el Equipo de Desarrollo entiende los elementos de la Lista del Producto al nivel necesario.
\end{itemize}

\begin{itemize}
    \item Expresar los items del \emph{Product Backlog}
\end{itemize}

\subsection{Equipo de Desarrollo}
El Equipo de Desarrollo está conformado por profesionales que realizan el trabajo de entregar un Incremento (ver Sección \ref{incremento}) de producto “Terminado” que potencialmente se puede poner en producción al final de cada Sprint (ver Sección \ref{sprint}). Cabe destacar que solo los miembros del Equipo de Desarrollo participan en la creación del Incremento, ellos se organizan y gestionan su propio trabajo. Los Equipos de Desarrollo tienen las siguientes características: 

\begin{itemize}
	\item Son autoorganizados. Nadie (ni siquiera el Scrum Master) indica al Equipo de Desarrollo cómo convertir elementos de la Lista del Producto en Incrementos de funcionalidad potencialmente desplegables.
	\item Los Equipos de Desarrollo son multifuncionales, esto es, como equipo cuentan con todas las habilidades necesarias para crear un Incremento de producto.
	\item Scrum no reconoce títulos para los miembros de un Equipo de Desarrollo independientemente del trabajo que realice cada persona
	\item Los Miembros individuales del Equipo de Desarrollo pueden tener habilidades especializadas y áreas en las que estén más enfocados, pero la responsabilidad recae en el Equipo de Desarrollo como un todo. 
\end{itemize}


\subsection{Scrum Master} \label{scrumMaster}

El Facilitador (o \emph{Scrum Master}) es el responsable de promover y apoyar Scrum como está definido en la Guía de Scrum \cite{scrumSchwaber}. Esto lo logran ayudando a todo el Equipo Scrum a entender la teoría, práctica, reglas y valores de Scrum \cite{scrumSchwaber}.

\section{Eventos de Scrum}

Para minimizar y regularizar la necesidad de reuniones no definidas en Scrum existen eventos predefinidos. Los eventos son bloques de tiempo (\textit{label-boxes}), de tal modo que todos tienen una duración máxima. En el caso de un Sprint (ver Sección \ref{sprint}), su duración es fija y no puede acortarse ni alargarse, el resto de eventos pueden terminar antes siempre y cuando se cumplan con el objetivo del evento \cite{scrumSchwaber}.


\subsection{Sprint} \label{sprint}
El Sprint es el corazón de Scrum, es un bloque de tiempo de aproximadamente un mes de duración durante el cual se crea un Incremento (ver Sección \ref{incremento}) del producto “Terminado” utilizable y potencialmente desplegable. Cada nuevo Sprint comienza inmediatamente después de la finalización del Sprint anterior \cite{scrumSchwaber}.

Los Sprint están compuestos por la Planificación del Sprint (\textit{Sprint Planning}), los Scrums Diarios (\textit{Daily Scrums}), el trabajo de desarrollo, la Revisión del Sprint (\textit{Sprint Review}), y la Retrospectiva del Sprint (\textit{Sprint Retrospective}). 

\section{Artefactos de Scrum}
Los artefactos de Scrum representan trabajo o valor en diversas formas que son útiles para proporcionar transparencia y oportunidades para la inspección y adaptación. Los artefactos definidos por Scrum están diseñados específicamente para maximizar la transparencia de la información clave, necesaria para asegurar que todos tengan el mismo entendimiento del artefacto \cite{scrumSchwaber}.

\subsection{Lista de Producto} \label{productBacklog}
La Lista de Producto es una lista ordenada de los objetivos y requisitos que son necesario en el producto. El Dueño de Producto (Product Owner) es el responsable de esta lista, incluyendo su contenido, disponibilidad y ordenación. La Lista de Producto va cambiando constantemente a medida de que el producto y el entorno en el que se usará también lo hacen, es decir cambia mientras se van identificando necesidades para que el producto pueda ser adecuado, competitivo y útil. La Lista de Producto enumera todas las características, funcionalidades, requisitos, mejoras y correcciones que constituyen cambios a realizarse sobre el producto para entregas futuras \cite{scrumSchwaber}.

\subsection{Lista de Pendientes del Sprint} \label{sprintBacklog}
La lista de Pendientes del Sprint es el conjunto de elementos seleccionados para realizar en el Sprint de la Lista de Productos, más un plan para entregar el Incremento (ver Sección \ref{incremento} del producto y cumplir los objetivos del Sprint \cite{scrumSchwaber}. Esta lista la realiza el Equipo de Desarrollo y se basa en la funcionalidad que formará parte del próximo Incremento.

\subsection{Incremento} \label{incremento}

El Incremento es la suma de todos los elementos de la Lista de Producto completados durante un Sprint y el valor de los incrementos de todos los Sprints anteriores. Al final de un Sprint el nuevo Incremento debe estar “Terminado”, lo cual significa que está en condiciones de ser utilizado y que cumple la Definición de “Terminado” del Equipo Scrum (cada equipo tiene su propia definición de un producto o Incremento terminado). El incremento es un paso hacia una visión o meta. El incremento debe estar en condiciones de utilizarse sin importar si el Dueño de Producto decide liberarlo o no \cite{scrumSchwaber}.


