\chapter{Marco Metodológico}
En este capítulo se explicará la metodología que se utilizó para el desarrollo del proyecto, que fue necesaria usar para cumplir con los objetivos planteados, conocida como Srum. A continuación se describirá el marco de desarrollo, las actividades y resultados de cada una de las etapas de esta metodología.

\section{¿Qué es Scrum?}
Scrum es un marco de trabajo para desarrollar, entregar y mantener productos complejos, en el cual las personas pueden abordar problemas complejos adaptativos, y a su vez entregar productos del máximo valor posible productiva y creativamente. Empezó a ser usado desde principios de los años 90. Scrum es un marco de trabajo dentro del cual se pueden emplear varios procesos y técnicas. El marco de trabajo Scrum consiste en los Equipos Scrum y sus roles, eventos, artefactos y reglas asociadas. \cite{scrumSchwaber}

Usando correctamente este marco de trabajo se puede mostrar la eficacia relativa de las técnicas de la gestión del producto y las técnicas de trabajo, de manera que a medida que va pasando el tiempo se pueda mejorar el producto, el equipo y el entorno de trabajo. Para que esto suceda es necesario que cada componente que interactúa dentro de Scrum cumpla con su propósito específico.

\section{Usos de Scrum}
Scrum inicialmente fue desarrollado para la gestión y desarrollo de productos. Se ha usado principalmente para:

\begin{enumerate}
	\item Investigar e identificar mercados viables, tecnologías y capacidades de productos.
	\item Desarrollar productos y mejoras.
	\item Liberar productos y mejoras tantas veces como sea posible durante el día.
	\item Desarrollar y mantener ambientes en la Nube (en línea, seguros, bajo demanda) y otros entornos operacionales para el uso de productos.
	\item Mantener y renovar productos.
\end{enumerate}

\section{Equipo de Scrum}

El equipo Scrum está conformado por el Dueño del Producto (\textit{Product Owner}), el Equipo de desarrollo (\textit{Development Team}) y un \textit{Scrum Master}. Los equipos Scrum son autoorganizados y multifuncionales. Los equipos autoorganizados eligen la mejor forma de llevar a cabo su trabajo y no son dirigidos por personas externas al equipo. Los equipos multifuncionales tienen todas las competencias necesarias para llevar a cabo el trabajo sin depender de otras personas que no son parte del equipo. El modelo de equipo en Scrum está diseñado para optimizar la flexibilidad, la creatividad y la productividad. \cite{scrumSchwaber}

\subsection{Dueño del Producto} \label{productOwner}
Es el responsable de maximizar el valor del producto resultante del trabajo del Equipo de Desarrollo. Es la única persona responsable de gestionar la Lista del Producto (\textit{Product Backlog}), esto incluye:
\begin{itemize}
\item Expresar claramente los elementos de la lista.
\item Ordenar los elementos de la lista, para alcanzar objetivos y misiones de manera eficiente.
	\item Optimizar el valor del trabajo que el Equipo de Desarrollo realiza.
	\item Asegurar que la lista sea visible, transparente y clara para todos y que muestra aquello en lo que el equipo trabajará a continuación.
	\item Asegurar que el Equipo de Desarrollo entiende los elementos de la Lista del Producto al nivel necesario.
\end{itemize}

\begin{itemize}
    \item Expresar los items del \emph{Product Backlog}
\end{itemize}

\subsection{Equipo de Desarrollo}
El Equipo de Desarrollo está conformado por profesionales que realizan el trabajo de entregar un Incremento (ver sección \ref{incremento}) de producto “Terminado” que potencialmente se puede poner en producción al final de cada Sprint (ver sección \ref{sprint}). Cabe destacar que solo los miembros del Equipo de Desarrollo participan en la creación del Incremento, ellos se organizan y gestionan su propio trabajo. Los Equipos de Desarrollo tienen las siguientes características: 

\begin{itemize}
	\item Son autoorganizados. Nadie (ni siquiera el Scrum Master) indica al Equipo de Desarrollo cómo convertir elementos de la Lista del Producto en Incrementos de funcionalidad potencialmente desplegables.
	\item Los Equipos de Desarrollo son multifuncionales, esto es, como equipo cuentan con todas las habilidades necesarias para crear un Incremento de producto.
	\item Scrum no reconoce títulos para los miembros de un Equipo de Desarrollo independientemente del trabajo que realice cada persona
	\item Los Miembros individuales del Equipo de Desarrollo pueden tener habilidades especializadas y áreas en las que estén más enfocados, pero la responsabilidad recae en el Equipo de Desarrollo como un todo. 
\end{itemize}


\subsection{Scrum Master} \label{scrumMaster}

El Facilitador (o \emph{Scrum Master}) es el responsable de promover y apoyar Scrum como está definido en la Guía de Scrum (ver \cite{scrumSchwaber}). Esto lo logran ayudando a todo el Equipo Scrum a entender la teoría, práctica, reglas y valores de Scrum. \cite{scrumSchwaber}

\section{Eventos de Scrum}
\subsection{La Iteración} \label{sprint}


\section{Artefactos de Scrum}
\subsection{Lista de Objetivos del Producto} \label{productBacklog}
La Lista de Objetivos del Producto es una lista ordenada de todas las tareas que deben ser completadas para entregar el producto. Esta lista nunca está completa, la primera versión muestra los primeros requerimientos conocidos y mejor entendidos. \cite{scrumSchwaber} La Lista de Objetivos del Producto evoluciona junto con el producto para actualizar los requerimientos dependiendo de las exigencias del cliente y del mercado. El Dueño del Producto (ver sección \ref{productOwner}) es el responsable de este artefacto.

\subsection{Lista de Tareas de la Iteración}
La Lista de Tareas de la Iteración es el conjunto de tareas de la lista de objetivos del producto seleccionadas para completar en una Iteración, puede verse como una predicción de lo que estará completado al llevarse a cabo la Iteración. \cite{scrumSchwaber} Esta lista va a ir cambiando a medida que el Equipo de Desarrollo identifique tareas y trabajo que debe ser realizado para completar los items durante el desarrollo de la Iteración, es decir, además de las tareas tomadas de la Lista de Objetivos del Producto, pueden surgir nuevas tareas que deben ser cumplidas para llevar a cabo el trabajo de la Iteración.

\subsection{Tablero de tareas}
El tablero de tareas es una tabla utilizada para gestionar el estado de los objetivos de la lista de objetivos del producto, contiene 4 columnas:

\begin{itemize}
   \item \textbf{Por hacer}: contiene las tareas que no se han empezado.
   \item \textbf{Haciendo}: contiene las tareas en las que se está trabajando actualmente.
   \item \textbf{Hechas}: contiene las tareas que se han completado.
   \item \textbf{Mejoras}: contiene tareas u objetivos que salen del alcance del proyecto actual y que pueden ser desarrolladas para una versión futura del producto.
\end{itemize}

Cada una de las tareas en este tablero puede ser asignada a una persona específica, también se pueden etiquetar dependiendo de la naturaleza de la tarea.

\subsection{Incremento} \label{incremento}


