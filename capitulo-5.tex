\chapter{Desarrollo} \label{development}
 

En este capítulo se presentan las actividades realizadas durante el desarrollo del proyecto correspondientes al período de pasantías Abril-Septiembre 2018 en la empresa iKêls Consulting \cite{ikels} siguiendo la metodología Scrum. En este proyecto se desarrolló la segunda versión de la aplicación web CPI, esta aplicación está conformada por tres módulos: 
\begin{enumerate}
    \item \underline{Módulo de administración de cadenas, tiendas, zonas, productos y categorías:} Este\\ módulo consistió en desarrollar la   interfaz gráfica para crear, editar y eliminar los registros de cada una de estas entidades, asignar tiendas a cadenas, asignar productos a categorías, y desarrollar la funcionalidad para listar los registros de productos con mecanismo de búsqueda. 
   \item \underline{Módulo de administración de precios:} Este módulo incluye la interfaz de creación, edición y eliminación de registros y el desarrollo de un mecanismo para importar precios en lote usando un archivo en formato Microsoft Excel.
    \item \underline{Módulo de generación de reportes:} Para este módulo se desarrolló la funcionalidad de generar reportes comparativos para:
   \begin{itemize}
       \item Gráfico de dispersión de precios / categorías.
       \item Gráfico de barras para un producto seleccionado separando precio normal y oferta.
       \item Gráfico tipo box-plot para el histórico de precios de un producto.
    \end{itemize}
    E implementar un mecanismo para exportar los reportes en un formato tipo tabla en Microsoft Excel.
\end{enumerate}
   El proyecto se dividió en 3 Fases :
\begin{itemize}
  \item Fase de Inducción, en la cual se hace la introducción del proyecto, el pasante se familiarizó con las herramientas y tecnologías necesarias para realizar el proyecto.
  \item Fase de Desarrollo la cual se dividió en 6 Sprints, en esta fase se hace la implementación de los tres módulos de la aplicación.
  \item Fase de Documentación en la cual se realiza parte de la documentación necesaria para la aplicación.
\end{itemize}
A continuación, se explicará en detalle en qué consiste cada Fase.

\section{Inducción}
Durante esta Fase el pasante realizó un proceso de familiarización con la empresa junto con el estudio de herramientas y lenguajes de trabajo que fueron empleados a lo largo del desarrollo del proyecto.
Realizó varios cursos tutoriales (Umbraco Course level 1, Learn ASP.NET MVC de Microsoft Virtual Academy, entre otros),  y utilizó recursos (guía de Umbraco) que la empresa proporcionó para tener los conocimientos necesarios para llevar a cabo el proyecto. Además investigó sobre herramientas como: ASP.NET, SQL Server, Highcharts y Visual Studio.

La duración de esta fase fue de una semana, sin embargo, el pasante tuvo que mantenerse en constante búsqueda de recursos e investigar sobre la marcha sobre herramientas o conceptos nuevos, los cuales se fueron necesitando en el camino. Todo este proceso fue sumamente útil y de gran aprendizaje ya que la mayoría de las herramientas utilizadas para el desarrollo del proyecto eran nuevas y muchos de los conceptos se investigaron a profundidad.

\section{Desarrollo}
La Fase de desarrollo de la aplicación se realizó en seis Sprints, los  primeros cinco tuvieron una duración en total de 17 semanas continuas distribuidas de la siguiente manera: 1er sprint 2 semanas, 2do sprint 3 semanas, 3er sprint 4 semanas, 4to sprint 4 semanas, 5to sprint 4 semanas y el último sprint se hizo luego de dos semanas de documentación y tuvo una duración de 3 semanas. A continuación una descripción del trabajo realizado en cada uno de ellos.
\pagebreak
\subsection{Primer Sprint}
Durante este Sprint se hizo el análisis y definición de los requerimientos de la aplicación, los cuales eran primero definir el alcance de la aplicación reflejados en los siguientes diagramas:


  \begin{figure}[H]
  \begin{center}
  \includegraphics[width=\textwidth]{cu_admin.png}
  \caption{Diegrama Diagrama de Casos de Uso para Admin inicial. Elaboración propia.}
  \label{fig:cu_admin_inicial}
  \end{center}
  \end{figure}

  \begin{figure}[H]
  \begin{center}
  \includegraphics[width=\textwidth]{cu_retailer_inicial.png}
  \caption{Diagrama de Casos de Uso para Retailer inicial. Elaboración propia.}
  \label{fig:cu_retailer_inicial}
  \end{center}
  \end{figure}
Definir la arquitectura a utilizar y por último la estructura de la base de datos como se ve reflejado en el diagrama ER de la Figura \ref{fig:er_nuevo}. El pasante inició la familiarización de la versión actual de CPI junto con el dueño del producto para resolver las posibles dudas de la funcionalidad, o conceptos necesarios para entender la aplicación. Además se creó el repositorio de Git, el proyecto de Visual Studio y el sitio de Umbraco, y por último el Dueño del producto sugirió dos posibles plantillas de HTML que podrían ser utilizadas para las vistas de la aplicación (Ver Figuras \ref{fig:adminator} y \ref{fig:vali}) y el pasante eligió la que era más adecuada visual y funcionalmente para la misma.

\begin{figure}[H]
  \begin{center}
  \includegraphics[width=\textwidth]{adminator.png}
  \caption{Primera opción para plantilla del front end de la aplicación (Admiator).}
  \label{fig:adminator}
  \end{center}
  \end{figure}

  \begin{figure}[H]
    \begin{center}
    \includegraphics[width=\textwidth]{vali.png}
    \caption{Segunda opción para plantilla del front end de la apliación (Vali). Esta fue la que se usó para la aplicación.}
    \label{fig:vali}
    \end{center}
    \end{figure}
\vskip 0.5cm
\pagebreak
\textbf{Actividades realizadas:}
\begin{itemize}
 \item Familiarización con la versión original de CPI. Esta actividad consistió principalmente en reuniones con el dueño del producto para aclarar dudas con respecto a los conceptos necesarios para entender la aplicación y sus funcionalidades.
 \item Elegir la arquitectura de la aplicación. El pasante junto con el Dueño del Producto decidieron utilizar el modelo de 4+1 Vistas de Philippe Kruchten \cite{vistasKruchten}, el cual  propone que un sistema software se ha de documentar y mostrar con 4 vistas bien diferenciadas y que se 
 tienen que relacionar entre sí con una vista más, que es la denominada vista “+1”. Estas vistas son: vista lógica, vista de procesos, vista de despliegue, vista física y la vista “+1” llamada vista de escenario.
 \item Definir los actores del sistema. En principio se definieron dos actores principales para la aplicación, \textit{Admin} (el administrador del sistema) y \textit{Retailer} (representantes de cadenas) ya que para el alcance que iba a tener esta segunda versión eran los únicos que la utilizarían.
 \item Realizar una matriz de permisología de las funcionalidades de la aplicación. Lo que el pasante realizó fue listar todas las posibles funciones de cada uno de los módulos de aplicación y definir cuáles eran los permisos que tendrán cada uno de los actores.
 \item Elegir las entidades de la aplicación. El pasante basándose en la primera versión de la aplicación y junto el dueño del producto tomaron la decisión de cuáles serían las entidades que se manejarían desde Umbraco (cadenas y zonas ) ya que al tener una cantidad de registros pequeños se pueden manejar comodamente como nodos (Ver Sección \ref{nodoUmbraco} )  del mismo  y el resto de las entidades al tener una gran cantidad de registros se dejaron en la base de datos de SQL Server (tiendas, productos, precios y catálogos). En la Figura \ref{fig:contenido} se muestran las entidades definidas en Umbraco.
 \begin{figure}[H]
  \begin{center}
  \includegraphics[scale=0.6]{content.png}
  \caption{Entidades cadenas y zonas en el contenido de Umbraco.}
  \label{fig:contenido}
  \end{center}
  \end{figure}
\pagebreak
 \item Rediseñar la base de datos. Luego de decidir cuáles eran las entidades a trabajar en la base de datos de SQL Server el pasante junto con el Dueño del Producto decidieron cuál podría ser la nueva estructura de la base de datos. A continuación se muestra el Diagrama de la versión original de CPI (ver Figura \ref{fig:er_viejo}) y el Diagrama ER de la versión 2 de CPI (ver Figura \ref{fig:er_nuevo})
 

  \begin{figure}[H]
  \begin{center}
  \includegraphics[width=\textwidth]{er.png}
  \caption{Diagrama de la base de datos de la nueva versión de CPI.}
  \label{fig:er_nuevo}
  \end{center}
  \end{figure}

  \begin{figure}[H]
    \begin{center}
    \includegraphics[width=\textwidth]{er_viejo.png}
    \caption{Diegrama ER de la base de datos de la versión 1 de CPI.}
    \label{fig:er_viejo}
    \end{center}
    \end{figure}
\pagebreak
 \item Crear el proyecto en Visual Studio.
 \item Crear el repositorio de Git. Para el proyecto se creó un repositorio llamado CPI el cual se utilizó para el control de versiones del proyecto. Para utilizar correctamente el repositorio el pasante tuvo que refrescar los comandos.
 \item Crear el sitio de Umbraco el cual consiste en descargar la versión de Umbraco a utilizar (7.10.4) y extraer los archivos en la carpeta del proyecto '/src/Umbraco.Site.CPI', luego crear el sitio en IIS (Internet Information Services) asignando los permisos necesarios sobre la carpeta, usando .Net 4.5 y colocar la ruta de acceso '/src/Umbraco.Site.CPI'.
 \item Elegir la plantilla a utilizar para las vistas de la aplicación. El pasante junto con el Dueño del Producto eligieron entre dos posibles plantillas, la elegida fue la que más se acopló a las necesidades de la aplicación.
 \item Investigar sobre Datatables ya que fue el pluggin (complemento) elegido por el Dueño del Producto para manejar dinámicamente las tablas.
 \item Listar las vistas que tendrá la aplicación, según los módulos que se eligieron como alcance inicial.
\end{itemize}

      
\textbf{Duración:} 2 semanas.


\subsection{Segundo Sprint (Módulo de Administración de cadenas, productos y catálogos)}
Durante este Sprint se implementó la funcionalidad para el manejo de las principales entidades de la aplicación: cadenas, productos y categorías, en la cual se desarrolló una interfaz gráfica en el back end de Umbraco usando un paquete llamado Fluidity para el manejo de estas entidades en la base de datos. Se definieron los Doctypes (tipos de documento) y Datatypes (tipos de datos) para crear el contenido en Umbraco. Se inició el desarrollo de las vistas de la aplicación. Y por último se empezó la implementación para la vista de listado de productos.
\vskip 0.5 cm
\textbf{Actividades realizadas:}
\begin{itemize}
  \item Definir los Doctypes para las entidades principales. El pasante definió las propiedades que tendrían cada una de las entidades que se encuentran en Umbraco, se muestran en la siguiente tabla:
  \begin{longtable}{ | p{5em} | l | l | }
    \hline
    \rowcolor{blue!25}
    \multicolumn{1}{|c|}{Doctype} &
    \multicolumn{1}{|c|}{Propiedades} &
    \multicolumn{1}{|c|}{Datatype} \\
    \hline
    \endhead
 
    \hline
    \endfoot
 
    \endlastfoot
 
    Cadena
        & Nombre & Textarea max150 \\
        \cline{2-3}
        & Abreviación & Textarea max65 \\
        \cline{2-3}
        & ID & Numeric \\
        \cline{2-3}
        & Correo electrónico & Email Address \\
        \cline{2-3}
        & Estado & On/off \\
        \cline{2-3}
        & Fecha actualización & Date Picker \\
    \hline
    Zona 
        & Nombre & Textarea max150 \\
        \cline{2-3}
        & ID & Numeric \\
        \cline{2-3}
        & Descripción & Email Richtext Editor \\
        \cline{2-3}
        & Disponible & On/off \\
    \hline
 
    \caption{Doctypes}
    \label{table:doctypes}
 \end{longtable}
 
  \item Definir los Datatypes que usan las entidades principales. El pasante creo tipos de datos necesarios para las propiedades de las entidades a usarse en Umbraco, se muestran en la siguiente tabla:
  \begin{longtable}{  l | l  }
    \hline\hline
    \rowcolor{blue!25}
    \textbf{Datatype} & \textbf{Función} \\
    \hline\hline
    \endhead
 
    \hline
    \endfoot
 
    \endlastfoot
 
    AbrvCompany & Cuadro de texto \\
    Attached images & Seleccionar una imagen \\
    Company & Seleccionar una cadena \\
    Catalog & Seleccionar un catálogo \\
    Email address & Introducir un correo electrónico \\
    Fechas & Seleccionar una fecha \\
    Name Company & Cuadro de texto \\
    Name Zone & Cuadro de texto \\
    On/off & Elegir una opción o no \\
    Products & Seleccionar un producto \\
    Textarea max150 & Introducir texto de máximo 150 caracteres \\
    Textarea max65 & Introducir texto de máximo 65 caracteres \\
    UPC & Seleccionar código UPC de un producto \\
    Zone  & Seleccionar una zona \\
    \hline
 
    \caption{Datatypes}
    \label{table:datatypes}
 \end{longtable}


  \item El Dueño del producto realizó una instalación local de la versión 1 de la aplicación CPI para que el pasante pudiera revisar a fondo la funcionalidad actual y además revisar el código fuente y la base de datos para ayudar al desarrollo de la nueva versión de la aplicación.
  \item El Dueño del Producto propuso dos paquetes de Umbraco para gestionar los datos de la base de datos de SQL Server (Fluidity o UI-Matic) y el pasante investigó sobre ambos y se decidió elegir Fluidity ya que entre ambos paquetes este era más recientemente, y se pueden manejar mayor cantidad de datos por lo que para el caso de CPI es mas conveniente el uso de esta herramienta ya que se manejan gran cantidad de registros de precios y productos,  y además era compatible con la versión de Umbraco utilizada para el desarrollo de la aplicación.
  
  Luego se desarrolló la interfaz de Fluidity para creación, edición y eliminación de registros de las entidades que se guardan en la base de datos de SQL Server, para de esta manera poder manejar los registros desde el back end de Umbraco.
  
  Para poder realizar la interfaz y poder manejar correctamente los datos de las tablas de la base de datos, en algunos casos se tuvo que agregar un campo adicional a la tabla y el mismo se tomaba como clave primaria, ya que estas tablas tenían dos claves primarias y Fluidity no manejaba correctamente este tipo de casos. En las Figuras \ref{fig:fluidity, , fig:fluidity_products, fig:fluidity_catalogs} se muestran la interfaz gráfica para la administración de entidades, administración de productos y administración de catálogos respectivamente.
  \begin{figure}[H]
    \begin{center}
    \includegraphics[scale=0.7]{fluidity.png}
    \caption{Interfaz de fluidity de las entidades.}
    \label{fig:fluidity}
    \end{center}
    \end{figure}
    
    \begin{figure}[H]
      \begin{center}
      \includegraphics[scale=0.7 ]{productos_fluidity.png}
      \caption{Interfaz de fluidity para administrar productos}
      \label{fig:fluidity_products}
      \end{center}
      \end{figure}


        \begin{figure}[H]
          \begin{center}
          \includegraphics[scale=0.7]{catalogos.png}
          \caption{Interfaz de fluidity para administrar catálogos.}
          \label{fig:fluidity_catalogs}
          \end{center}
          \end{figure}
  \pagebreak
  \item Se desarrollaron las principales partials views (vistas parciales) que tienen en común todas las vistas de la aplicación, como por ejemplo la barra de navegación (Ver Figura \ref{fig:navbar})y la barra lateral (Ver Figura \ref{fig:sidebar}).
 
  \begin{figure}[H]
    \begin{center}
    \includegraphics[width=\textwidth]{navbar.png}
    \caption{Vista parcial Barra de Navegación.}
    \label{fig:navbar}
    \end{center}
    \end{figure}

    \begin{figure}[H]
      \begin{center}
      \includegraphics[scale=0.6]{sidebar.png}
      \caption{Vista parcial Barra Lateral.}
      \label{fig:sidebar}
      \end{center}
      \end{figure}

  \item Se realizó la implementación de la vista y la funcionalidad para listar los productos por cadena, y esto se hizo utilizando Datatables.
  \begin{figure}[H]
    \begin{center}
    \includegraphics[width=\textwidth]{productos.png}
    \caption{Vista de la lista de productos del front end.}
    \label{fig:products}
    \end{center}
    \end{figure}

\end{itemize}
\textbf{Duración:} 3 semanas.


\subsection{Tercer Sprint (Primera parte del Módulo de Generación de Reportes)}
En este Sprint se desarrolló parte de la funcionalidad de generación de reportes comparativos. Este módulo cuenta con tres tipos de reportes, en este Sprint se desarrollaron dos de ellos (reporte de comparación de precios de un producto en distintas cadenas, y reporte histórico de precios de un producto). 

Ya que el tiempo de desarrollo de los gráficos usando la herramienta Highcharts tomó más tiempo del que se tenía planificado porque al ser una herramienta que el pasante no utilizó anteriormente se necesitó de tiempo para aprender a utilizarla y luego para poder traer los datos correctamente de la base de datos (los cuales son los que se ven reflejados en cada uno de los gráficos) se necesitó de la traducción, entendimiento y mejora considerable del código de la versión antigua y esto se hizo bastante engorroso ya que los scripts no estaban bien documentados.
\vskip 0.5cm
\textbf{Actividades realizadas:}
\begin{itemize}
  \item Elección de la herramienta para hacer los gráficos necesarios para los reportes. El Dueño del Producto propuso dos herramientas para la generación de los gráficos, la primera era Chartjs que es de licencia gratuita y la segunda se llama Highcharts la cual es de licencia paga. 
  
  El pasante para realizar la elección de la herramienta, realizó ejemplos de gráficos usando ambas opciones para comparar complejidad, facilidad de conseguir recursos y ejemplos para ayudar con el desarrollo, entre otras cosas, y llegó a la conclusión que la herramienta más apta para ser usada en la aplicación era Highcharts ya que contaba con todos los tipos de gráficos necesarios para la aplicación (Chartjs no contaba con gráfico boxplot) y cuenta con mayor documentación y ejemplos en los que se podía apoyar para el desarrollo de los gráficos.
  \item Revisión de la documentación y la Referencia API de JavaScript de Highcharts para poder comenzar con el desarrollo de los gráficos para los reportes.
  \item Implementación del reporte de comparación de precios de un producto en común entre distintas cadenas (Ver Figura \ref{fig:barras}). Este reporte se divide en dos partes, la primera es un gráfico de barras en el que se ven reflejados el precio normal y el precio por oferta del producto en las distintas cadenas, y la segunda parte es una tabla con información más detallada del producto (precio PVP, precio oferta, IVA del producto y número de tiendas dentro de la cadena en la que se encuentra disponible dicho producto).
  
  \item Implementación del reporte histórico de un producto en un período de tiempo, para esto se realizó un gráfico boxplot en el que se muestra una caja por fecha en el que esté registrado el precio de ese producto dentro del período elegido.
\end{itemize}

\begin{figure}[H]
  \begin{center}
  \includegraphics[width=\textwidth]{reporte_barras.png}
  \caption{Vista de un reporte de precios de un producto.}
  \label{fig:barras}
  \end{center}
  \end{figure}

\textbf{Duración:} 4 semanas.

\subsection{Cuarto Sprint (Segunda parte del Módulo de Generación de Reportes)}

En este Sprint se realizó el reporte de precios de productos por catálogos (gráfico de dispersión), y se corrigieron detalles de los otros dos reportes ya implementados.
\vskip 0.5cm
\textbf{Actividades realizadas:}
\begin{itemize}
  \item Se agregó a la interfaz de Fluidity que ya se encontraba implementada la función para asignar productos a catálogos como se muestra en la Figura \ref{fig:prod_cat}.
  \begin{figure}[H]
    \begin{center}
    \includegraphics[scale=0.7]{asignar_producto.png}
    \caption{Interfaz de Fluidity para asignar Producto a un Catálogo.}
    \label{fig:prod_cat}
    \end{center}
    \end{figure}
  
  \item Agregar enlace de la tabla de productos al reporte de comparación de precios. A la lista de todos los productos se le agregó un enlace desde el código SKU del producto hacia el reporte de comparación de precios de productos de diferentes cadenas.
  \item Implementación del reporte de precios de productos por categorías. Primero para comenzar con la implementación se realizó un boceto con datos ficticios del gráfico de dispersión para que el pasante se familiarizara con la complejidad de este tipo de gráficos. Luego se realizó la traducción, mejora y entendimiento del código de la versión antigua de la aplicación para manejar los datos que se encuentran en la base de datos y reflejarlos en el gráfico.
  \item Se inició la construcción del diccionario de Umbraco con las palabras estáticas y que se repiten en cada una de las vistas de la aplicación.
  \item Se corrigieron detalles de los reportes anteriores. Para el reporte de comparación de precios de un producto se corrigió el formato de la fecha (pasó  de utilizarse “mm/dd/yyyy” a el formato ISO “yyyymmdd”) de igual manera para el reporte histórico, se arreglaron detalles del código de HTML y eficiencia de los controladores para traer la información de la base de datos.
\end{itemize}

\textbf{Duración:} 4 semanas.


\subsection{Quinto Sprint (Módulo de Administración de Precios y Corrección de estructura del proyecto)}
En este Sprint se implementó el módulo de administración de precios de productos. Este módulo consiste en la creación de la interfaz gráfica para la creación, edición y eliminación de registros, e importar precios por lote mediante un archivo en formato Microsoft Excel, durante el sprint el pasante se dió cuenta de que había un error en la estructura de la solución del proyecto, ya que la misma no estaba siguiendo el patrón MVC que se debía tener en la solución de Visual Studio por lo que parte del Sprint se dedicó a corregir esto. Este error en la estructura fue debido a la falta de comunicación entre el pasante y el Dueño del Producto sobre aspectos técnicos del proyecto.
\vskip 0.5cm
\textbf{Actividades realizadas:}
\begin{itemize}
  \item Se implementó la autenticación, inicio y cierre de sesión para el front end de la aplicación (Ver Figura \ref{fig:login}). Para esto primero se definen los permisos de usuarios, en este caso, para el alcance de la aplicación únicamente se tiene el usuario \emph{retailer}, y se creó un grupo de miembro en el back end de Umbraco en la sección Members llamado \emph{cpi user}, luego se desarrolló la vista para el iniciar y cerrar sesión.
  \begin{figure}[H]
    \begin{center}
    \includegraphics[width=\textwidth]{login.png}
    \caption{Vista Login de la aplicación.}
    \label{fig:login}
    \end{center}
    \end{figure}

  \item Se desarrolló la interfaz gráfica para creación, edición y eliminación de registros usando el paquete de Umbraco Fluidity para poder manejar esta entidad directo de la plataforma, modificando directamente la base de datos de SQL Server.
  \begin{figure}[H]
    \begin{center}
    \includegraphics[scale=0.6]{precios.png}
    \caption{Interfaz de fluidity para administración de precios.}
    \label{fig:prices}
    \end{center}
    \end{figure}
  \item Se realizó la implementación para importar por lote los precios desde un archivo en formato Microsoft Excel.
  \item El código de la aplicación pasó a revisión y de esta manera encontrar detalles de escritura y aplicar buenas prácticas de desarrollo. Por lo que se mejoraron nombres y declaraciones de variables, se agregaron mas comentarios para el entendimiento del código, y eliminación de librerías que no se usaban por las clases.
  \item Reorganización de controladores y vistas de la aplicación. Luego de revisar y corregir los errores encontrados, se consiguió un error en la estructura de la solución de Visual Studio, ya que no se estaba siguiendo el patrón MVC que se había escogido al principio de la pasantía. Por lo que para resolver esto, se organizaron los controladores y las vistas de la aplicación.
\end{itemize}

\textbf{Duración:} 4 semanas.

\subsection{Sexto Sprint (Completar funcionalidades faltantes en los módulos)}
Este es el último Sprint de desarrollo, en el mismo se implementaron funcionalidades de los módulos que no se pudieron completar en los Sprints pasados, para el primer módulo se completaron las entidades restantes tiendas y zonas, y para el módulo de reportes se hizo la exportación en formato Microsoft Excel.
\vskip 0.5cm
\textbf{Actividades realizadas:}
\begin{itemize}
  \item Agregar Datatypes necesarios para la entidad zonas (como por ejemplo NameZone que es para colocar el nombre de la zona que tiene máximo 30 caracteres).
  \item Crear el Doctype de zonas para administrar las zonas que se usarán en la aplicación.
  \item Hacer la interfaz gráfica en el back end de umbraco usando fluidity para administrar las tiendas de la base de datos de SQL Server. Y además se pueden asignar tiendas a cadenas. La Figura \ref{fig:fluidity_tiendas} muestra la interfaz gráfica para administrar tiendas.
  \begin{figure}[H]
    \begin{center}
    \includegraphics[scale=0.5]{tiendas.png}
    \caption{Interfaz de fluidity para adinistrar tiendas.}
    \label{fig:fluidity_tiendas}
    \end{center}
    \end{figure}


  \item Para el módulo de generación de reportes se desarrolló la funcionalidad para exportar los reportes en un archivo formato Microsoft Excel.
\end{itemize}

\textbf{Duración:} 3 semanas.

\vskip 0.5cm


\section{Documentación} \label{documentation}
Esta última fase del proyecto duró las últimas dos semanas del período de pasantía. Se realizó la documentación necesaria para la aplicación, se dejó expresado el estado final del sistema, se elaboraron los documentos de instalación de CPI y el Documento de Arquitectura de Software (ver Anexo \ref{das}) donde se detallan los componentes y casos de uso desarrollados de la aplicación.




