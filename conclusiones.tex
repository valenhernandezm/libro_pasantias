\chapter*{Conclusiones y Recomendaciones}
Durante este proyecto de pasantía se desarrolló una nueva versión de la aplicación web CPI que permite realizar comparaciones de precios entre productos ofrecidos por diferentes cadenas de ventas al detal. El proceso de desarrollo requirió realizar un proceso completo de re-ingeniería para adaptar la versión antigua a plataformas y modelos de gestión más modernos. A continuación se presentan las conclusiones y recomendaciones más importantes una vez completado este proyecto de pasantías.

\section{Conclusiones}
Las conclusiones a continuación resumen los resultados y hallazgos más relevantes obtenidos durante el proyecto de pasantía
\begin{itemize}
	\item El módulo de administración de cadenas, productos y categorías permite al usuario manejar a su gusto cada una de estas entidades, además de listar todos los productos disponibles en la cadena al que pertenece el mismo.
	\item El módulo de generación de reportes permite al usuario de la cadena realizar comparaciones de los productos de las otras cadenas, con el fin de poder confeccionar una estructura de precios que resulte simultáneamente atractiva para atraer a sus clientes así como rentable para mantener el negocio a flote. 
	\item El módulo de administración de precios permite al usuario manejar a su gusto los precios de los productos pertenecientes a esta cadena.
	\item El uso de una aplicación Web brinda un servicio rápido de respuesta y accesibilidad desde cualquier parte de la internet. Además ofrece una interfaz amigable y fácil de usar.
	\item En el desarrollo se presentaron retrasos debido a que primero a pesar de que hubo una fase de inducción del pasante a los conceptos y herramientas básicas necesarias para la implementación de la aplicación, una semana no fue suficiente para prepararlo para abarcar y cumplir con el alcance inicial del proyecto, ya que a medida de que se fue avanzando en el desarrollo de los módulos se tuvo la necesidad de realizar investigaciones o tutoriales para completar el conocimiento que hacía falta para completar los mismos, y trató de resolver todas las dudas por cuenta propia y no se apoyó o buscó ayuda con alguno de los integrantes del equipo de desarrollo, lo que muchas veces hizo que se retrasara el trabajo. Segundo a pesar de que el pasante trató de tener al día los conocimientos para cumplir con el plan de trabajo, faltó comunicación con el tutor académico y dueño del producto para resolver dudas necesarias para el desarrollo de la aplicación por lo que se tuvo una equivocación en la estructura de la solución de Visual Studio por lo que se tuvo que tomar tiempo que estaba planificado para el desarrollo de otras funcionalidades para resolver este tipo de inconvenientes.
	\item La documentación y las guías de usuario son dos componentes muy importantes en el desarrollo de cualquier sistema ya que permiten, para los usuarios manejar de manera sencilla el sistema y para los futuros administradores y desarrolladores realizar el mantenimiento y las expansiones que correspondan. Por lo que es muy importante que se vayan realizando a medida que se va avanzando el desarrollo de la aplicación, y no dejarlo para el final ya que puede que queden detalles importantes sin quedar plasmados en los documentos.
	\item Finalmente, se puede concluir que gracias al presente trabajo de pasantía el pasante obtuvo una gran cantidad de enseñanzas y crecimiento como futuro profesional que complementaron y reforzaron los conocimientos logrados en la Universidad. Fue una oportunidad inigualable para poner en práctica una gran cantidad de conceptos y teorías aprendidas durante la estadía de la misa, así como para aprender que la práctica es muy distinta a la teoría.
\end{itemize}

\section{Recomendaciones}
A continuación se desglosan las recomendaciones sobre aquellos aspectos que podrían mejorarse:
\begin{itemize}
	\item Se recomienda finalizar la implementación de las funcionalidades de los módulos que no se alcanzaron a desarrollar durante la pasantía. Y a su vez empezar el desarrollo de los módulos que no eran parte del alcance inicial con el fin de tener la nueva versión de CPI completa para lanzarla a producción sustituyendo a la antigua versión.
 	\item Es recomendable realizar la documentación del sistema al mismo momento que se va avanzando con su desarrollo, a manera de realizarla con calidad, completitud y correctitud.
	\item Es importante que los desarrolladores se mantengan en constante contacto tanto con los actores de la aplicación como con el dueño del producto para asegurarse de que las necesidades están siendo cumplidas, así como de realizar las correcciones que puedan surgir a tiempo.
\end{itemize}
