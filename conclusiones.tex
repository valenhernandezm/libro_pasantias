\chapter*{Conclusiones y Recomendaciones}
La fase de ejecución del período de pasantía representó un complemento primordial para la educación y crecimiento profesional del pasante. Las pasantías permiten al estudiante experimentar en el campo laboral aplicando los conocimientos adquiridos durante la carrera y a enfrentarse con un ambiente totalmente nuevo en el cual es necesario adquirir nuevos conocimientos y en muchos casos aprender de nuevas herramientas que le permitan cumplir con el objetivo planteado. 

Por otro lado, se introduce al estudiante en el contexto laboral el cual es fundamental para su desempeño en un futuro y  permite obtener una visión sobre cuáles deben ser las actitudes que se debe tomar en una organización o empresa.

Con respecto al desarrollo de la aplicación fue un constante reto para el pasante ya que la mayoría de las herramientas eran nuevas y debió aprender muchas cosas desde cero, empezando por conocer la plataforma Umbraco que aunque es amigable la curva de aprendizaje es bastante pronunciada y toma tiempo internalizar y entender todos los conceptos necesarios para crear un sitio en Umbraco.

El mayor reto fue el aprovechar correctamente el modelo MVC en ASP.NET ya que aunque es un concepto en el que se trabajó durante toda la carrera, aplicarlo sobre este marco en un principio no se hizo de la manera correcta ya que se mezclaron los controladores con las vistas y luego de corregir el error realmente se pudieron separar los diferentes aspectos de la aplicación (lógica de entrada, lógica de negocios y lógica de la interfaz de usuario), y a su vez proporcionar un vago acoplamiento entre estos elementos, lo que facilitó la administración de la complejidad para compilar la aplicación, ya que permite centrarse en cada momento en un único aspecto de la implementación. 

Una vez culminado el período de pasantías y al cumplir con todas las actividades expuestas anteriormente en la empresa iKêls Consulting \cite{ikels}, se puede afirmar que este tiempo ha sido provechoso para todos los entes involucrados ya que ahora la empresa cuenta con una versión más reciente de la aplicación web CPI que les permite a los usuarios de la misma comparar precios de los productos y generar reportes para estudiar el mercado y cuál es el comportamiento de los diferentes productos en las tiendas de venta al detal, y para el pasante resultó ser una fuente de gran conocimiento ya que se enfrentó obstáculos y pruebas que con la ayuda del tutor industrial, académico e integrantes de la empresa pudo superarlos y cumplir con todos los objetivos.    


\section{Recomendaciones}
A continuación se desglosan las recomendaciones sobre aquellos aspectos que podrían mejorarse:
\begin{itemize}
	\item Se recomienda continuar con la implementación de la aplicación desarrollando los módulos que no eran parte del alcance inicial con el fin de tener la nueva versión de CPI completa para lanzarla a producción sustituyendo a la antigua versión y que los usuarios puedan tener el beneficio completo que presenta la misma para el estudio de los precios de productos y saber aprovecharlos para ser de mayor competencia en el mercado.
 	\item Es recomendable realizar la documentación del sistema al mismo momento que se va avanzando con su desarrollo, a manera de realizarla con calidad, completitud y correctitud.
	\item Se recomienda a los desarrolladores del sistema mantener un continuo seguimiento de los avances del proyecto junto con el Dueño del Producto para verificar que todos los requerimientos del mismo están siendo cumplidos satisfactoriamente.
\end{itemize}
